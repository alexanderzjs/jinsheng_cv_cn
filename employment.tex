\makerubrichead{工作经历}

\noindent\makefield{\faBookmark}{蚂蚁金服(阿里巴巴) 技术专家 (P7) 安全多方计算组区块链团队 \hfill 2018年七月 -- 2019年四月}\\

\smallskip
\setlength\parindent{18pt}\makefield{\faStar}{参与并负责设计中国移动隐私保护查询项目}\\
\setlength\parindent{36pt}\makefield{\faChevronRight}{调研并设计改进的PSI(隐私保护的集合查询)技术以支撑移动的基本需求}\\
\setlength\parindent{36pt}\makefield{\faChevronRight}{参与开发改进的PSI(隐私保护的集合查询)技术实现亿级别数据查询}\\
\setlength\parindent{18pt}\makefield{\faStar}{参与并负责设计20多家银行、基金、消金和互金的隐私保护逻辑回归模型LR建模方案}\\
\setlength\parindent{36pt}\makefield{\faChevronRight}{设计秘密共享技术实现进行隐私保护LR建模}\\		
\setlength\parindent{36pt}\makefield{\faChevronRight}{改进设计的隐私保护建模方案来支持万级别数据和千级别特征的隐私保护LR建模}\\
\setlength\parindent{36pt}\makefield{\faChevronRight}{引入混淆电路技术进一步改进LR建模方案以实现分钟级别隐私保护LR建模}\\
\setlength\parindent{18pt}\makefield{\faStar}{参与并负责设计20多家银行、基金、消金和互金的隐私保护树模型GBDT预测方案}\\
\setlength\parindent{36pt}\makefield{\faChevronRight}{设计基于混淆电路技术实现的隐私保护GBDT树模型预测方案}\\		
\setlength\parindent{36pt}\makefield{\faChevronRight}{独立开发实现混淆电路以及相关核心技术}\\		
\setlength\parindent{36pt}\makefield{\faChevronRight}{提出并设计基于混淆电路和同态加密的隐私保护GBDT树模型预测方案}\\		
\setlength\parindent{18pt}\makefield{\faStar}{设计多方安全计算编译器}\\
\setlength\parindent{36pt}\makefield{\faChevronRight}{设计基于混淆电路和秘密共享技术的编译器}\\		
\setlength\parindent{36pt}\makefield{\faChevronRight}{设计基于混淆电路和秘密共享技术的虚拟机来完成编译内容的运行}\\		
\smallskip

\vspace{0.2cm}
\noindent\makefield{\faBookmark}{Ebay中级软件工程师, 信任和身份管理组 \hfill 2016年五月 -- 2018年七月}\\
\smallskip
\setlength\parindent{18pt}\makefield{\faStar}{实现基于消息推送的双重认证机制}\\
\setlength\parindent{36pt}\makefield{\faChevronRight}{利用Ebay的FIDO服务来实现消息推送认证机制}\\
\setlength\parindent{18pt}\makefield{\faStar}{实现跨平台单点认证模型来支持任意设备的无密码访问}\\
\setlength\parindent{18pt}\makefield{\faStar}{设计并实现已有登录系统到REST服务的迁移来明确登录系统中各服务的分离性和易维护性}\\
\setlength\parindent{36pt}\makefield{\faChevronRight}{深入研究现有登录系统的各个服务以及不同用户类型的不同服务分支的区别}\\
\setlength\parindent{36pt}\makefield{\faChevronRight}{实现不同类别用户(10个类别)的Cookie管理(一共80多个Cookie)}\\
\setlength\parindent{18pt}\makefield{\faStar}{实现一个商用标准的REST服务,该服务用于追踪消费者登陆行为并提供数据给其他数据分析组使用}\\
\smallskip

\vspace{0.2cm}
\noindent\makefield{\faBookmark}{Nok Nok Labs软件工程师实习 \hfill 2015年一月 -- 2015年十二月}\\
\smallskip
\setlength\parindent{18pt}\makefield{\faStar}{设计并整合IBM Tivoli服务器与已有的多重认证服务器}\\
\setlength\parindent{18pt}\makefield{\faStar}{实现商用化多租户的访问控制并整合已有的多重认证服务器}\\
\setlength\parindent{18pt}\makefield{\faStar}{实现多重认证服务器的用户邮件注册功能}\\
\smallskip

\vspace{0.4cm}
\noindent\makefield{\faBookmark}{Iowa State University计算机科学系首席助教 \hfill 2013年九月 -- 2014年十二月}\\
\smallskip
\setlength\parindent{18pt}\makefield{\faStar}{COM S 106: 网页编程导论 (主讲教授: Dr. Susan (Shu-Hui) Chang)}\\
\setlength\parindent{36pt}\makefield{\faChevronRight}{参与课程讲义编制、课程作业设计以及课程项目的设计}\\
\smallskip

\vspace{0.2cm}
\noindent\makefield{\faBookmark}{Iowa State University计算机科学系首席助教 \hfill 2012年一月 -- 2013年九月}\\
\smallskip
\setlength\parindent{18pt}\makefield{\faStar}{COM S 103: 计算机应用导论 (主讲教授: Dr. Susan (Shu-Hui) Chang)}\\
\setlength\parindent{36pt}\makefield{\faChevronRight}{参与学生答疑,课程作业的以及课程项目的批改}\\
\smallskip

\vspace{0.2cm}
\noindent\makefield{\faBookmark}{Iowa State University计算机科学系助教 \hfill 2012年一月 -- 2012年五月}\\
\smallskip
\setlength\parindent{18pt}\makefield{\faStar}{COM S 552: 操作系统原理 (主讲教授: Dr. Wensheng Zhang)}\\
\setlength\parindent{36pt}\makefield{\faChevronRight}{参与学生答疑,课程作业的以及课程项目的批改}\\
\smallskip

\vspace{0.4cm}
\noindent\makefield{\faBookmark}{Iowa State University计算机科学系助研 \hfill 2010年九月 -- 2011年十二月}\\
\smallskip
\setlength\parindent{18pt}\makefield{\faStar}{在助研阶段,我和我的导师Dr. Wensheng Zhang (计算机科学系) 以及Dr. Daji Qiao (计算机电气工程系)紧密合作来完成博士学业。我的科研方向是应用密码学,计算机安全/隐私保护,包括相关算法,系统,数据库,网络的设计以及更多方向的研究。}\\
\smallskip

\vspace{0.2cm}
\noindent\makefield{\faBookmark}{中科院信息安全国家重点实验室助研 \hfill 2009年六月 -- 2009年八月}\\
\smallskip
\setlength\parindent{18pt}\makefield{\faStar}{在实习阶段,我和我的导师Dr. Yuqing Zhang紧密合作来完成大学生研究计划课题“研究RepTrap攻击对信任管理系统的影响”。}\\
\setlength\parindent{0pt}